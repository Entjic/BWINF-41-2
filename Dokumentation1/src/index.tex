\maketitle
\tableofcontents

\vspace{0.5cm}


\section{Lösungsidee}\label{sec:losungsidee}
Das Problem der Aufgabenstellung beschreibt eine Variante des NP-vollständigen Traveling-Salesman Problems.
Es ist somit nicht möglich optimale Lösungen in polynomieller Zeit für beliebige Graphen zu finden.
Für die Graphen aus den Beispieldateien ist dies aufgrund ihrer Größe auch nicht möglich.
Folglich müssen zur Bewältigung der Aufgabe Heuristiken herangezogen werden. \\
Zunächst wird eine initiale Lösung mittels der Nearest-Neighbour Heuristik ermittelt.
Diese wird danach iterativ durch die TwoOpt-Postoptimierung verbessert.

\subsection{Nearest-Neighbour Heuristik}\label{subsec:nearest-neighbour-heuristik}
1.
Zunächst wird ein Startknoten \textit{v}\textsubscript{Start} ermittelt.
Ausgehend von \textit{v}\textsubscript{Start}, wird der Knoten \textit{v}\textsubscript{0} mit der geringsten Distanz gewählt.
\textit{v}\textsubscript{Start} und \textit{v}\textsubscript{0} werden als besucht markiert.
Die Kante \textit{e} zwischen diesen beiden Knoten wird für die Suche des nächsten Knotens genutzt.

2.
Solange es noch unbesuchte Knoten gibt, wird folgendes Verfahren wiederholt:
Betrachte in aufsteigender Distanz zu \textit{v}\textsubscript{0} alle Knoten.
Wenn die Kante, die den aktuell betrachteten Knoten und \textit{v}\textsubscript{0} verbindet \textit{e}
in einem stumpfen Winkel schneidet
Wähle unter allen nicht besuchten Knoten denjenigen Knoten \textit{v}, der die geringste Entfernung zu \textit{v}\textsubscript{0}
hat und gleichzeitig dessen Kante zu \textit{v}\textsubscript{0} \textit{e} in einem stumpfen Winkel schneidet (Winkelbedingung).

Ist es nicht möglich einen solchen Knoten zu finden, muss einen Schritt zurückgegangen werden und der

\subsection{TwoOpt Postoptimierung}\label{subsec:twoopt-postoptimierung}


\section{Umsetzung}\label{sec:umsetzung}
Die Implementierung erfolgt in Java 17.


\section{Beispiele}\label{sec:beispiele}
Genügend Beispiele einbinden!
Die Beispiele von der BwInf-Webseite sollten hier diskutiert werden, aber auch eigene Beispiele sind
sehr gut – besonders wenn sie Spezialfälle abdecken.
Aber bitte nicht 30 Seiten Programmausgabe hier einfügen!


\section{Quellcode}\label{sec:quellcode}
Unwichtige Teile des Programms sollen hier nicht abgedruckt werden.
Dieser Teil sollte nicht mehr als 2–3 Seiten umfassen, maximal 10.

