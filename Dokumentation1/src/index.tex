\maketitle
\tableofcontents

\vspace{0.5cm}

\textbf{Anleitung:} Trage oben in den Zeilen 8 bis 10 die Aufgabennummer, die Teilnahme-ID und die/den Bearbeiterin/Bearbeiter dieser Aufgabe mit Vor- und Nachnamen ein.
Vergiss nicht, auch den Aufgabennamen anzupassen (statt "`\LaTeX-Dokument"')!

Dann kannst du dieses Dokument mit deiner \LaTeX-Umgebung übersetzen.

Die Texte, die hier bereits stehen, geben ein paar Hinweise zur Einsendung.
Du solltest sie aber in deiner Einsendung wieder entfernen!


\section{Lösungsidee}\label{sec:losungsidee}
Die Idee der Lösung sollte hieraus vollkommen ersichtlich werden, ohne dass auf die eigentliche Implementierung Bezug genommen wird.


\section{Umsetzung}\label{sec:umsetzung}
Hello\cite{Hierholzer}, world.


\section{Beispiele}\label{sec:beispiele}
Genügend Beispiele einbinden!
Die Beispiele von der BwInf-Webseite sollten hier diskutiert werden, aber auch eigene Beispiele sind
sehr gut – besonders wenn sie Spezialfälle abdecken.
Aber bitte nicht 30 Seiten Programmausgabe hier einfügen!


\section{Quellcode}\label{sec:quellcode}
Unwichtige Teile des Programms sollen hier nicht abgedruckt werden.
Dieser Teil sollte nicht mehr als 2–3 Seiten umfassen, maximal 10.

