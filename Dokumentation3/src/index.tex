\maketitle
\tableofcontents

\vspace{0.5cm}


\section{Lösungsidee}\label{sec:losungsidee}
Um Pfandkuchenstapel zu sortieren, wird die Technik des Dynamic Programmings verwendet.
Idee ist es, das Problem, also den zu sortierenden Pfandkuchenstapel, in identische Teilprobleme zu zerlegen.
Im Kontext der Aufgabe bedeutet das, an allen möglichen Indizes die Wende-und-Essoperation beginnen zu lassen.
Der jeweilige Index der Wende-und-Essoperation wird gemerkt.
Dieses Verfahren wird so lange wiederholt, bis der Pfandkuchenstapel sortiert ist.
Die möglichen Abfolgen von Wende-und-Essoperationen zur Sortierung eines Pfandkuchenstapels werden gesammelt
und abschlie{\ss}end derjenige gewählt, der die wenigsten Wende-und-Essoperationen benötigt.
Danach wird der aktuelle Pfandkuchenstapel und die jeweilige beste gefundene Lösung gemerkt.
Zu Beginn eines jeden Rekursionsschritts wird überprüft, ob nicht genau dieser Pfandkuchenstapel bereits gelöst wurde.
Ist der Pfandkuchenstapel bereits gelöst worden, so wird darauf verzichtet diesen wieder in alle möglichen Pfandkuchenstapel
aufzuteilen und es wird direkt die Zugfolge des gelösten Pfandkuchenstapels verwendet. \\
Dieses Verfahren garantiert einerseits, dass die Zugfolge optimal kurz ist, und zugleich nicht unnötig
oft derselbe Pfandkuchenstapel gelöst wird. \\
Um die PWUE-Zahl einer Pfandkuchenstapelhöhe zu generieren, müssen alle Pfandkuchenstapel der besagten Höhe generiert,
sortiert und abschlie{\ss}end bezüglich der Länge ihrer Zugfolge evaluiert werden.
Problematisch hierbei ist die schiere Menge der zu prüfenden Pfandkuchenstapel, sowie die Menge der Einträge, die
infolge des Dynamic Programming Ansatzes, gemerkt werden müssen. \\
Für die Anzahl der zu prüfenden Pfandkuchenstapel gilt $h!$ wobei $h$ die Höhe der betrachteten Pfandkuchenstapel ist.
Um die zu merkende Datenmenge zu reduzieren, wird die Art und Weise wie Pfandkuchenstapel gelöst werden modifiziert.
Ziel ist es nun nicht mehr, die Zugfolge zu finden, sondern lediglich die Anzahl der nötigen Wende-und-Essoperationen.
Zudem werden die PWUE-Zahlen dynamisch generiert.
Sucht man also bspw.\ die PWUE-Zahl der Höhe 7, so wird zunächst die PWUE-Zahl der Höhe 1, dann 2, dann 3, etc.\ generiert,
und dementsprechend auch alle Pfandkuchenstapel der Höhe 1, 2, 3 etc.
Dieser Bottom-Up Ansatz hat den Vorteil, dass man sich sicher sein kann, dass alle Pfandkuchenstapel der vorherigen
Höhen bereits gelöst wurden.
Das eröffnet die Möglichkeit Einträge gezielt nicht zu merken.
Im vorher verwendeten Top-Down Ansatz würde man bei fehlenden Einträgen davon ausgehen, dass diese schlicht noch nicht
berechnet wurden und deswegen fehlen.
Jetzt kann aber davon ausgegangen werden, dass diese absichtlich fehlen,
schlie{\ss}lich wurden ja alle vorherigen Pfandkuchenstapel gelöst.
Dadurch kann durch die \textit{Absenz von Einträgen} Information ausgedrückt werden, ohne kostbaren Speicher aufzuwenden.
Konkret werden, nachdem die PWUE-Zahl einer Zahl $n$ ermittelt wurde, alle Einträge gelöscht, welche eine so kurze Zugfolge
aufweisen, sodass sie für die PWUE-Zahl von $n + 1$ nicht relevant sind.
Die Zugfolge ist genau dann zu kurz, wenn sie kleiner als $n - 2$ ist.
Beim Lösen eines Pfandkuchenstapels der Höhe $n$ werden nun alle Pfandkuchenstapel erzeugt,
welche durch eine einzige Wende-und-Essoperation erreichbar sind.
Diese haben demnach die Höhe $n - 1$.
Ist einer der resultierenden Pfandkuchenstapel nicht gemerkt, so kann davon ausgegangen werden, dass der ursprüngliche
Pfandkuchen auch nicht relevant für die PWUE-Betrachtung ist, weil er genau einen Zug
mehr braucht als ein Stapel der weniger als $PWUE(n - 1) - 2$ Operationen benötigt.
Der ursprüngliche Pfandkuchen braucht in dem Fall also maximal $PWUE(n - 1) - 2$ Operationen um sortiert zu werden.
Da $PWUE(n + 1) >= PWUE(n)$ muss $PWUE(n) >= PWUE(n - 1)$ gelten.
Um einen Einfluss auf $PWUE(n)$ zu nehmen muss die Anzahl


\section{Umsetzung}\label{sec:umsetzung}
Die Implementierung erfolgt in Java 17.


\section{Beispiele}\label{sec:beispiele}

\subsection{Beispiel Dateien}\label{subsec:beispiel-dateien}
Es folgen die Pfandkuchenstapel der Beispieldateien und die jeweilige Programmausgabe.
\subsubsection{pancake0.txt}
{\obeylines

Wende-und-Essoperation an Index 0: [1, 5, 4, 2, 3] -> [3, 2, 4, 5]
    Wende-und-Essoperation an Index 0: [3, 2, 4, 5] -> [5, 4, 2]
    Ursprünglicher Pfandkuchenstapel -> Sortierter Pfandkuchenstapel -> Indizes benötigter Operationen
    [1, 5, 4, 2, 3]
    [5, 4, 2]
    [0, 0]
    Benötigte Wende-und-Essoperationen 2
    \subsubsection{pancake1.txt}

    Wende-und-Essoperation an Index 1: [5, 2, 4, 7, 1, 3, 6] -> [5, 6, 3, 1, 7, 4]
    Wende-und-Essoperation an Index 1: [5, 6, 3, 1, 7, 4] -> [5, 4, 7, 1, 3]
    Wende-und-Essoperation an Index 2: [5, 4, 7, 1, 3] -> [5, 4, 3, 1]
    Ursprünglicher Pfandkuchenstapel -> Sortierter Pfandkuchenstapel -> Indizes benötigter Operationen
    [5, 2, 4, 7, 1, 3, 6]
    [5, 4, 3, 1]
    [1, 1, 2]
    Benötigte Wende-und-Essoperationen 3
    \subsubsection{pancake2.txt}

    Wende-und-Essoperation an Index 4: [2, 4, 6, 3, 5, 7, 1, 8] -> [2, 4, 6, 3, 8, 1, 7]
    Wende-und-Essoperation an Index 5: [2, 4, 6, 3, 8, 1, 7] -> [2, 4, 6, 3, 8, 7]
    Wende-und-Essoperation an Index 3: [2, 4, 6, 3, 8, 7] -> [2, 4, 6, 7, 8]
    Wende-und-Essoperation an Index 0: [2, 4, 6, 7, 8] -> [8, 7, 6, 4]
    Ursprünglicher Pfandkuchenstapel -> Sortierter Pfandkuchenstapel -> Indizes benötigter Operationen
    [2, 4, 6, 3, 5, 7, 1, 8]
    [8, 7, 6, 4]
    [4, 5, 3, 0]
    Benötigte Wende-und-Essoperationen 4
    \subsubsection{pancake3.txt}

    Wende-und-Essoperation an Index 0: [6, 3, 7, 9, 2, 8, 4, 11, 1, 10, 5] -> [5, 10, 1, 11, 4, 8, 2, 9, 7, 3]
    Wende-und-Essoperation an Index 5: [5, 10, 1, 11, 4, 8, 2, 9, 7, 3] -> [5, 10, 1, 11, 4, 3, 7, 9, 2]
    Wende-und-Essoperation an Index 8: [5, 10, 1, 11, 4, 3, 7, 9, 2] -> [5, 10, 1, 11, 4, 3, 7, 9]
    Wende-und-Essoperation an Index 2: [5, 10, 1, 11, 4, 3, 7, 9] -> [5, 10, 9, 7, 3, 4, 11]
    Wende-und-Essoperation an Index 0: [5, 10, 9, 7, 3, 4, 11] -> [11, 4, 3, 7, 9, 10]
    Wende-und-Essoperation an Index 1: [11, 4, 3, 7, 9, 10] -> [11, 10, 9, 7, 3]
    Ursprünglicher Pfandkuchenstapel -> Sortierter Pfandkuchenstapel -> Indizes benötigter Operationen
    [6, 3, 7, 9, 2, 8, 4, 11, 1, 10, 5]
    [11, 10, 9, 7, 3]
    [0, 5, 8, 2, 0, 1]
    Benötigte Wende-und-Essoperationen 6
    \subsubsection{pancake4.txt}

    Wende-und-Essoperation an Index 1: [2, 8, 3, 9, 12, 13, 1, 6, 10, 5, 11, 4, 7] -> [2, 7, 4, 11, 5, 10, 6, 1, 13, 12, 9, 3]
    Wende-und-Essoperation an Index 7: [2, 7, 4, 11, 5, 10, 6, 1, 13, 12, 9, 3] -> [2, 7, 4, 11, 5, 10, 6, 3, 9, 12, 13]
    Wende-und-Essoperation an Index 2: [2, 7, 4, 11, 5, 10, 6, 3, 9, 12, 13] -> [2, 7, 13, 12, 9, 3, 6, 10, 5, 11]
    Wende-und-Essoperation an Index 1: [2, 7, 13, 12, 9, 3, 6, 10, 5, 11] -> [2, 11, 5, 10, 6, 3, 9, 12, 13]
    Wende-und-Essoperation an Index 0: [2, 11, 5, 10, 6, 3, 9, 12, 13] -> [13, 12, 9, 3, 6, 10, 5, 11]
    Wende-und-Essoperation an Index 6: [13, 12, 9, 3, 6, 10, 5, 11] -> [13, 12, 9, 3, 6, 10, 11]
    Wende-und-Essoperation an Index 2: [13, 12, 9, 3, 6, 10, 11] -> [13, 12, 11, 10, 6, 3]
    Ursprünglicher Pfandkuchenstapel -> Sortierter Pfandkuchenstapel -> Indizes benötigter Operationen
    [2, 8, 3, 9, 12, 13, 1, 6, 10, 5, 11, 4, 7]
    [13, 12, 11, 10, 6, 3]
    [1, 7, 2, 1, 0, 6, 2]
    Benötigte Wende-und-Essoperationen 7
    \subsubsection{pancake5.txt}

    Wende-und-Essoperation an Index 13: [11, 5, 6, 12, 1, 14, 9, 7, 3, 2, 8, 10, 13, 4] -> [11, 5, 6, 12, 1, 14, 9, 7, 3, 2, 8, 10, 13]
    Wende-und-Essoperation an Index 0: [11, 5, 6, 12, 1, 14, 9, 7, 3, 2, 8, 10, 13] -> [13, 10, 8, 2, 3, 7, 9, 14, 1, 12, 6, 5]
    Wende-und-Essoperation an Index 5: [13, 10, 8, 2, 3, 7, 9, 14, 1, 12, 6, 5] -> [13, 10, 8, 2, 3, 5, 6, 12, 1, 14, 9]
    Wende-und-Essoperation an Index 2: [13, 10, 8, 2, 3, 5, 6, 12, 1, 14, 9] -> [13, 10, 9, 14, 1, 12, 6, 5, 3, 2]
    Wende-und-Essoperation an Index 5: [13, 10, 9, 14, 1, 12, 6, 5, 3, 2] -> [13, 10, 9, 14, 1, 2, 3, 5, 6]
    Wende-und-Essoperation an Index 3: [13, 10, 9, 14, 1, 2, 3, 5, 6] -> [13, 10, 9, 6, 5, 3, 2, 1]
    Ursprünglicher Pfandkuchenstapel -> Sortierter Pfandkuchenstapel -> Indizes benötigter Operationen
    [11, 5, 6, 12, 1, 14, 9, 7, 3, 2, 8, 10, 13, 4]
    [13, 10, 9, 6, 5, 3, 2, 1]
    [13, 0, 5, 2, 5, 3]
    Benötigte Wende-und-Essoperationen 6
    \subsubsection{pancake6.txt}

    Wende-und-Essoperation an Index 1: [6, 10, 5, 9, 3, 11, 7, 15, 1, 2, 13, 12, 4, 8, 14] -> [6, 14, 8, 4, 12, 13, 2, 1, 15, 7, 11, 3, 9, 5]
    Wende-und-Essoperation an Index 4: [6, 14, 8, 4, 12, 13, 2, 1, 15, 7, 11, 3, 9, 5] -> [6, 14, 8, 4, 5, 9, 3, 11, 7, 15, 1, 2, 13]
    Wende-und-Essoperation an Index 6: [6, 14, 8, 4, 5, 9, 3, 11, 7, 15, 1, 2, 13] -> [6, 14, 8, 4, 5, 9, 13, 2, 1, 15, 7, 11]
    Wende-und-Essoperation an Index 2: [6, 14, 8, 4, 5, 9, 13, 2, 1, 15, 7, 11] -> [6, 14, 11, 7, 15, 1, 2, 13, 9, 5, 4]
    Wende-und-Essoperation an Index 7: [6, 14, 11, 7, 15, 1, 2, 13, 9, 5, 4] -> [6, 14, 11, 7, 15, 1, 2, 4, 5, 9]
    Wende-und-Essoperation an Index 3: [6, 14, 11, 7, 15, 1, 2, 4, 5, 9] -> [6, 14, 11, 9, 5, 4, 2, 1, 15]
    Wende-und-Essoperation an Index 0: [6, 14, 11, 9, 5, 4, 2, 1, 15] -> [15, 1, 2, 4, 5, 9, 11, 14]
    Wende-und-Essoperation an Index 0: [15, 1, 2, 4, 5, 9, 11, 14] -> [14, 11, 9, 5, 4, 2, 1]
    Ursprünglicher Pfandkuchenstapel -> Sortierter Pfandkuchenstapel -> Indizes benötigter Operationen
    [6, 10, 5, 9, 3, 11, 7, 15, 1, 2, 13, 12, 4, 8, 14]
    [14, 11, 9, 5, 4, 2, 1]
    [1, 4, 6, 2, 7, 3, 0, 0]
    Benötigte Wende-und-Essoperationen 8
    \subsubsection{pancake7.txt}

    Wende-und-Essoperation an Index 5: [11, 16, 14, 1, 9, 12, 4, 2, 6, 13, 7, 3, 15, 10, 5, 8] -> [11, 16, 14, 1, 9, 8, 5, 10, 15, 3, 7, 13, 6, 2, 4]
    Wende-und-Essoperation an Index 0: [11, 16, 14, 1, 9, 8, 5, 10, 15, 3, 7, 13, 6, 2, 4] -> [4, 2, 6, 13, 7, 3, 15, 10, 5, 8, 9, 1, 14, 16]
    Wende-und-Essoperation an Index 0: [4, 2, 6, 13, 7, 3, 15, 10, 5, 8, 9, 1, 14, 16] -> [16, 14, 1, 9, 8, 5, 10, 15, 3, 7, 13, 6, 2]
    Wende-und-Essoperation an Index 8: [16, 14, 1, 9, 8, 5, 10, 15, 3, 7, 13, 6, 2] -> [16, 14, 1, 9, 8, 5, 10, 15, 2, 6, 13, 7]
    Wende-und-Essoperation an Index 10: [16, 14, 1, 9, 8, 5, 10, 15, 2, 6, 13, 7] -> [16, 14, 1, 9, 8, 5, 10, 15, 2, 6, 7]
    Wende-und-Essoperation an Index 5: [16, 14, 1, 9, 8, 5, 10, 15, 2, 6, 7] -> [16, 14, 1, 9, 8, 7, 6, 2, 15, 10]
    Wende-und-Essoperation an Index 2: [16, 14, 1, 9, 8, 7, 6, 2, 15, 10] -> [16, 14, 10, 15, 2, 6, 7, 8, 9]
    Wende-und-Essoperation an Index 3: [16, 14, 10, 15, 2, 6, 7, 8, 9] -> [16, 14, 10, 9, 8, 7, 6, 2]
    Ursprünglicher Pfandkuchenstapel -> Sortierter Pfandkuchenstapel -> Indizes benötigter Operationen
    [11, 16, 14, 1, 9, 12, 4, 2, 6, 13, 7, 3, 15, 10, 5, 8]
    [16, 14, 10, 9, 8, 7, 6, 2]
    [5, 0, 0, 8, 10, 5, 2, 3]
    Benötigte Wende-und-Essoperationen 8
    \subsection{Weitere Pfandkuchenstapel}\label{sec:weitere-pfandkuchenstapel}
    \subsubsection{Höhe 18}

    Wende-und-Essoperation an Index 3: [17, 10, 8, 14, 12, 6, 11, 5, 4, 9, 18, 2, 1, 15, 7, 3, 16, 13] -> [17, 10, 8, 13, 16, 3, 7, 15, 1, 2, 18, 9, 4, 5, 11, 6, 12]
    Wende-und-Essoperation an Index 1: [17, 10, 8, 13, 16, 3, 7, 15, 1, 2, 18, 9, 4, 5, 11, 6, 12] -> [17, 12, 6, 11, 5, 4, 9, 18, 2, 1, 15, 7, 3, 16, 13, 8]
    Wende-und-Essoperation an Index 12: [17, 12, 6, 11, 5, 4, 9, 18, 2, 1, 15, 7, 3, 16, 13, 8] -> [17, 12, 6, 11, 5, 4, 9, 18, 2, 1, 15, 7, 8, 13, 16]
    Wende-und-Essoperation an Index 7: [17, 12, 6, 11, 5, 4, 9, 18, 2, 1, 15, 7, 8, 13, 16] -> [17, 12, 6, 11, 5, 4, 9, 16, 13, 8, 7, 15, 1, 2]
    Wende-und-Essoperation an Index 6: [17, 12, 6, 11, 5, 4, 9, 16, 13, 8, 7, 15, 1, 2] -> [17, 12, 6, 11, 5, 4, 2, 1, 15, 7, 8, 13, 16]
    Wende-und-Essoperation an Index 1: [17, 12, 6, 11, 5, 4, 2, 1, 15, 7, 8, 13, 16] -> [17, 16, 13, 8, 7, 15, 1, 2, 4, 5, 11, 6]
    Wende-und-Essoperation an Index 10: [17, 16, 13, 8, 7, 15, 1, 2, 4, 5, 11, 6] -> [17, 16, 13, 8, 7, 15, 1, 2, 4, 5, 6]
    Wende-und-Essoperation an Index 5: [17, 16, 13, 8, 7, 15, 1, 2, 4, 5, 6] -> [17, 16, 13, 8, 7, 6, 5, 4, 2, 1]
    Ursprünglicher Pfandkuchenstapel -> Sortierter Pfandkuchenstapel -> Indizes benötigter Operationen
    [17, 10, 8, 14, 12, 6, 11, 5, 4, 9, 18, 2, 1, 15, 7, 3, 16, 13]
    [17, 16, 13, 8, 7, 6, 5, 4, 2, 1]
    [3, 1, 12, 7, 6, 1, 10, 5]
    Benötigte Wende-und-Essoperationen 8
    \subsection{PWUE-Zahlen}\label{sec:pwue-zahlen}
    \subsubsection{Höhe 1}

    PWUE of number 1 is 0
    Pfandkuchenstapel -> Sortierter Pfandkuchenstapel -> Indizes benötigter Operationen
Example Worstcase Pancakestack
    [1]
    [1]
    []
    Sorter map contains 0 entries
    Timings report
    Time spend finding PWUE nr 2 ms
    \subsubsection{Höhe 2}

    PWUE of number 2 is 1
    Pfandkuchenstapel -> Sortierter Pfandkuchenstapel -> Indizes benötigter Operationen
Example Worstcase Pancakestack
    [1, 2]
    [2]
    [0]
    Sorter map contains 1 entries
    Timings report
    Time spend finding PWUE nr 1 ms
    \subsubsection{Höhe 3}

    PWUE of number 3 is 2
    Pfandkuchenstapel -> Sortierter Pfandkuchenstapel -> Indizes benötigter Operationen
Example Worstcase Pancakestack
    [1, 3, 2]
    [1]
    [1, 1]
    Sorter map contains 6 entries
    Timings report
    Time spend finding PWUE nr 6 ms
    \subsubsection{Höhe 4}

    PWUE of number 4 is 2
    Pfandkuchenstapel -> Sortierter Pfandkuchenstapel -> Indizes benötigter Operationen
Example Worstcase Pancakestack
    [2, 4, 1, 3]
    [4, 1]
    [0, 0]
    Sorter map contains 5 entries
    Timings report
    Time spend finding PWUE nr 13 ms
    \subsubsection{Höhe 5}

    PWUE of number 5 is 3
    Pfandkuchenstapel -> Sortierter Pfandkuchenstapel -> Indizes benötigter Operationen
Example Worstcase Pancakestack
    [1, 4, 5, 2, 3]
    [4, 3]
    [2, 0, 0]
    Sorter map contains 16 entries
    Timings report
    Time spend finding PWUE nr 19 ms
    \subsubsection{Höhe 6}

    PWUE of number 6 is 3
    Pfandkuchenstapel -> Sortierter Pfandkuchenstapel -> Indizes benötigter Operationen
Example Worstcase Pancakestack
    [4, 1, 6, 2, 5, 3]
    [5, 3, 1]
    [3, 0, 2]
    Sorter map contains 24 entries
    Timings report
    Time spend finding PWUE nr 94 ms
    \subsubsection{Höhe 7}

    PWUE of number 7 is 4
    Pfandkuchenstapel -> Sortierter Pfandkuchenstapel -> Indizes benötigter Operationen
Example Worstcase Pancakestack
    [4, 6, 7, 5, 1, 3, 2]
    [7, 5, 2]
    [1, 0, 3, 2]
    Sorter map contains 86 entries
    Timings report
    Time spend finding PWUE nr 49 ms
    \subsubsection{Höhe 8}

    PWUE of number 8 is 5
    Pfandkuchenstapel -> Sortierter Pfandkuchenstapel -> Indizes benötigter Operationen
Example Worstcase Pancakestack
    [2, 4, 8, 1, 5, 7, 3, 6]
    [8, 4, 3]
    [5, 4, 0, 2, 0]
    Sorter map contains 2666 entries
    Timings report
    Time spend finding PWUE nr 176 ms
    \subsubsection{Höhe 9}

    PWUE of number 9 is 5
    Pfandkuchenstapel -> Sortierter Pfandkuchenstapel -> Indizes benötigter Operationen
Example Worstcase Pancakestack
    [3, 7, 1, 9, 2, 6, 5, 8, 4]
    [6, 5, 4, 1]
    [1, 2, 4, 0, 2]
    Sorter map contains 1705 entries
    Timings report
    Time spend finding PWUE nr 1123 ms
    \subsubsection{Höhe 10}

    PWUE of number 10 is 6
    Pfandkuchenstapel -> Sortierter Pfandkuchenstapel -> Indizes benötigter Operationen
Example Worstcase Pancakestack
    [5, 10, 1, 4, 9, 3, 6, 8, 2, 7]
    [5, 4, 3, 2]
    [2, 1, 2, 4, 5, 2]
    Example Worstcase Pancakestack
    [1, 6, 9, 3, 7, 2, 10, 4, 8, 5]
    [10, 8, 7, 6]
    [3, 3, 6, 4, 0, 3]
    Example Worstcase Pancakestack
    [3, 2, 8, 5, 9, 1, 6, 10, 4, 7]
    [5, 4, 2, 1]
    [2, 0, 4, 5, 1, 3]
    Example Worstcase Pancakestack
    [1, 5, 8, 4, 10, 2, 6, 9, 3, 7]
    [8, 7, 6, 2]
    [1, 3, 2, 5, 0, 1]
    Example Worstcase Pancakestack
    [2, 6, 10, 1, 7, 4, 9, 3, 8, 5]
    [5, 4, 3, 1]
    [0, 7, 1, 5, 3, 1]
    Example Worstcase Pancakestack
    [1, 5, 10, 6, 8, 4, 7, 3, 9, 2]
    [10, 8, 4, 2]
    [1, 0, 5, 1, 1, 3]
    Example Worstcase Pancakestack
    [2, 6, 10, 5, 9, 4, 7, 3, 8, 1]
    [10, 9, 4, 1]
    [1, 0, 5, 1, 1, 3]
    Example Worstcase Pancakestack
    [4, 6, 8, 3, 7, 5, 10, 2, 9, 1]
    [10, 7, 3, 1]
    [2, 3, 5, 0, 1, 1]
    Example Worstcase Pancakestack
    [2, 5, 10, 3, 8, 1, 6, 9, 4, 7]
    [9, 7, 3, 1]
    [1, 6, 6, 2, 0, 1]
    Example Worstcase Pancakestack
    [5, 3, 7, 2, 6, 9, 1, 8, 10, 4]
    [10, 8, 7, 2]
    [4, 4, 0, 2, 2, 2]
    Sorter map contains 254385 entries
    Timings report
    Time spend finding PWUE nr 9721 ms
    \subsubsection{Höhe 11}

    PWUE of number 11 is 6
    Pfandkuchenstapel -> Sortierter Pfandkuchenstapel -> Indizes benötigter Operationen
Example Worstcase Pancakestack
    [3, 9, 7, 1, 10, 4, 8, 5, 11, 2, 6]
    [9, 8, 4, 2, 1]
    [0, 0, 7, 3, 4, 1]
    Example Worstcase Pancakestack
    [5, 11, 3, 10, 1, 8, 9, 6, 2, 7, 4]
    [11, 10, 8, 7, 2]
    [0, 0, 5, 1, 5, 1]
    Example Worstcase Pancakestack
    [3, 5, 11, 8, 2, 10, 7, 4, 1, 9, 6]
    [11, 10, 7, 4, 1]
    [1, 0, 7, 2, 1, 5]
    Example Worstcase Pancakestack
    [3, 8, 4, 1, 7, 2, 5, 9, 11, 6, 10]
    [11, 10, 8, 5, 2]
    [2, 3, 3, 0, 3, 0]
    Example Worstcase Pancakestack
    [3, 6, 1, 9, 7, 2, 10, 4, 8, 11, 5]
    [11, 8, 4, 2, 1]
    [0, 4, 6, 0, 0, 3]
    Example Worstcase Pancakestack
    [6, 8, 11, 4, 5, 1, 7, 2, 10, 3, 9]
    [11, 8, 5, 3, 2]
    [0, 7, 0, 3, 6, 4]
    Example Worstcase Pancakestack
    [5, 9, 4, 10, 2, 7, 11, 1, 8, 3, 6]
    [11, 9, 4, 3, 2]
    [0, 5, 2, 7, 0, 3]
    Example Worstcase Pancakestack
    [2, 1, 11, 10, 8, 4, 7, 3, 6, 9, 5]
    [11, 10, 8, 6, 3]
    [0, 1, 5, 0, 6, 0]
    Example Worstcase Pancakestack
    [3, 4, 10, 7, 1, 9, 5, 8, 2, 6, 11]
    [11, 10, 8, 5, 1]
    [0, 2, 2, 1, 2, 4]
    Example Worstcase Pancakestack
    [2, 4, 9, 11, 3, 6, 8, 1, 7, 5, 10]
    [10, 5, 4, 3, 1]
    [0, 5, 7, 4, 5, 2]
    Sorter map contains 49110 entries
    Timings report
    Time spend finding PWUE nr 129081 ms
    \subsubsection{Höhe 12}

    PWUE of number 12 is 7
    Pfandkuchenstapel -> Sortierter Pfandkuchenstapel -> Indizes benötigter Operationen
Example Worstcase Pancakestack
    [4, 9, 5, 11, 8, 3, 10, 6, 2, 7, 12, 1]
    [12, 11, 10, 6, 2]
    [0, 0, 4, 3, 0, 6, 1]
    Example Worstcase Pancakestack
    [2, 12, 3, 8, 11, 7, 4, 9, 6, 1, 10, 5]
    [12, 11, 9, 6, 1]
    [3, 8, 9, 3, 0, 5, 0]
    Example Worstcase Pancakestack
    [8, 5, 1, 11, 2, 7, 12, 3, 6, 10, 4, 9]
    [11, 10, 9, 7, 3]
    [2, 0, 1, 7, 3, 1, 2]
    Example Worstcase Pancakestack
    [7, 3, 12, 4, 9, 5, 11, 1, 6, 10, 2, 8]
    [11, 10, 9, 4, 3]
    [0, 4, 0, 6, 1, 2, 4]
    Example Worstcase Pancakestack
    [4, 7, 1, 9, 5, 12, 8, 2, 10, 3, 11, 6]
    [12, 9, 7, 6, 3]
    [2, 9, 6, 3, 3, 0, 2]
    Example Worstcase Pancakestack
    [1, 8, 4, 9, 6, 11, 3, 7, 10, 2, 12, 5]
    [12, 11, 10, 7, 4]
    [0, 0, 5, 3, 5, 0, 4]
    Example Worstcase Pancakestack
    [1, 7, 9, 4, 11, 5, 8, 3, 12, 6, 10, 2]
    [12, 6, 5, 4, 2]
    [1, 2, 6, 4, 4, 0, 4]
    Example Worstcase Pancakestack
    [5, 7, 11, 4, 9, 3, 12, 6, 8, 1, 10, 2]
    [12, 10, 9, 8, 7]
    [0, 4, 2, 5, 0, 4, 0]
    Example Worstcase Pancakestack
    [2, 12, 3, 10, 4, 6, 11, 5, 8, 1, 9, 7]
    [7, 5, 4, 3, 1]
    [0, 3, 7, 3, 6, 3, 1]
    Example Worstcase Pancakestack
    [3, 6, 11, 4, 7, 9, 1, 8, 10, 2, 5, 12]
    [12, 10, 8, 7, 4]
    [0, 1, 5, 5, 1, 3, 3]
    Sorter map contains 2512306 entries
    Timings report
    Time spend finding PWUE nr 1303151 ms
    \subsubsection{Höhe 13}

    todo
}


\section{Quellcode}\label{sec:quellcode}
Unwichtige Teile des Programms sollen hier nicht abgedruckt werden.
Dieser Teil sollte nicht mehr als 2–3 Seiten umfassen, maximal 10.

